\documentclass{article}
\usepackage[utf8]{inputenc}
\usepackage[spanish]{babel}
\usepackage{listings}
\usepackage{graphicx}
\graphicspath{ {images/} }
\usepackage{cite}

\begin{document}

\begin{titlepage}
    \begin{center}
        \vspace*{1cm}
            
        \Huge
        \textbf{Proyecto Final}
            
        \vspace{0.5cm}
        \LARGE
        Los primeros pasos
            
        \vspace{1.5cm}
            
        \textbf{Juan Pablo Gómez Giraldo}
            
        \vfill
        
        Informática II
            
        \vspace{0.8cm}
            
        \Large
        Departamento de Ingeniería Electrónica y Telecomunicaciones\\
        Universidad de Antioquia\\
        Medellín\\
        Marzo de 2021
            
    \end{center}
\end{titlepage}

\newpage
\section{Idea Videojuego}\label{idea}
Videojuego de plataformas en dos dimensiones, en el que el personaje principal es un dinosaurio, con traje de astronauta, el cual avanza por el espacio exterior en busca de sus huevos.

El dinosaurio debe avanzar hasta rescatar sus huevos, pero debe esquivar meteoritos, que destruirán toda su especie.

En su camino, el dinosaurio encontrará algunos agujeros, que podrán ser rojos o azules. Estos agujeros funcionan de igual manera que en el famoso juego de mesa 'escalera', en este caso, el agujero azul lo ayudará a seguir hacia adelante un poco más rápido, acortando el camino, y el agujero rojo lo devolverá hasta cierta parte del juego.

Nuestro dinosaurio tendrá un tiempo limite para esquivar todo lo que tiene en su en torno y llegar hasta sus huevos a salvo.


\bibliographystyle{IEEEtran}
\bibliography{references}

\end{document}
